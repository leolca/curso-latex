\begin{frame}[fragile]
\frametitle{Documento}
\framesubtitle{Fórmulas}
  \begin{verbatim}
\usepackage{amsmath}
ou
\usepackage{mathtools}
  \end{verbatim}

  Como inserir fórmulas?
  \begin{itemize}
  \item \begin{verbatim}
\( ... \)  ou  $ ... $
        \end{verbatim}
  \item \begin{verbatim}
\begin{equation} ... \end{equation}
        \end{verbatim}
  \end{itemize}
\end{frame}


\begin{frame}[fragile]
\frametitle{Documento}
\framesubtitle{Fórmulas}
  \scriptsize
  \begin{columns}[c]
  \column{.5\textwidth}
  \begin{verbatim}
\begin{equation}
\forall x \in X, 
\quad \exists y \leq \epsilon
\end{equation}
  \end{verbatim}
  \column{.5\textwidth}
  \begin{fmpage}{\textwidth}
\begin{equation}
\forall x \in X, 
\quad \exists y \leq \epsilon
\end{equation}
  \end{fmpage}
  \end{columns}

  \begin{columns}[c]
  \column{.5\textwidth}
  \begin{verbatim}
\begin{equation}
\alpha, \beta, \gamma, \delta,
\epsilon, \zeta, \eta, \theta,
\Gamma, \Delta, \Theta, \Lambda
\pi, \Pi, \phi, \Phi
\end{equation}
  \end{verbatim}
  \column{.5\textwidth}
  \begin{fmpage}{\textwidth}
\begin{equation}
\alpha, \beta, \gamma, \delta,
\epsilon, \zeta, \eta, \theta,
\Gamma, \Delta, \Theta, \Lambda
\pi, \Pi, \phi, \Phi
\end{equation}
  \end{fmpage}
  \end{columns}


  \begin{columns}[c]
  \column{.5\textwidth}
  \begin{verbatim}
\begin{equation}
\cos (2\theta) = 
\cos^2 \theta - \sin^2 \theta
\end{equation}
  \end{verbatim}
  \column{.5\textwidth}
  \begin{fmpage}{\textwidth}
\begin{equation}
\cos (2\theta) = 
\cos^2 \theta - \sin^2 \theta
\end{equation}
  \end{fmpage}
  \end{columns}
\end{frame}


\begin{frame}[fragile]
\frametitle{Documento}
\framesubtitle{Fórmulas}
  \scriptsize
  \begin{columns}[c]
  \column{.5\textwidth}
  \begin{verbatim}
\begin{equation}
\lim_{x \to \infty} \exp(-x) = 0
\end{equation}
  \end{verbatim}
  \column{.5\textwidth}
  \begin{fmpage}{\textwidth}
\begin{equation}
\lim_{x \to \infty} \exp(-x) = 0
\end{equation}
  \end{fmpage}
  \end{columns}


  \begin{columns}[c]
  \column{.5\textwidth}
  \begin{verbatim}
\begin{equation}
a \bmod b
\end{equation}
  \end{verbatim}
  \column{.5\textwidth}
  \begin{fmpage}{\textwidth}
\begin{equation}
a \bmod b
\end{equation}
  \end{fmpage}
  \end{columns}


  \begin{columns}[c]
  \column{.5\textwidth}
  \begin{verbatim}
\begin{equation}
x \equiv a \pmod b
\end{equation}
  \end{verbatim}
  \column{.5\textwidth}
  \begin{fmpage}{\textwidth}
\begin{equation}
x \equiv a \pmod b
\end{equation}
  \end{fmpage}
  \end{columns}
\end{frame}


\begin{frame}[fragile]
\frametitle{Documento}
\framesubtitle{Fórmulas}
  \scriptsize
  \begin{columns}[c]
  \column{.5\textwidth}
  \begin{verbatim}
\begin{equation}
k_{n+1} = n^2 + k_n^2 - k_{n-1}
\end{equation}
  \end{verbatim}
  \column{.5\textwidth}
  \begin{fmpage}{\textwidth}
\begin{equation}
k_{n+1} = n^2 + k_n^2 - k_{n-1}
\end{equation}
  \end{fmpage}
  \end{columns}


  \begin{columns}[c]
  \column{.5\textwidth}
  \begin{verbatim}
\begin{equation}
f(n) = n^5 + 4n^2 + 2 |_{n=17}
\end{equation}
  \end{verbatim}
  \column{.5\textwidth}
  \begin{fmpage}{\textwidth}
\begin{equation}
f(n) = n^5 + 4n^2 + 2 |_{n=17}
\end{equation}
  \end{fmpage}
  \end{columns}


  \begin{columns}[c]
  \column{.5\textwidth}
  \begin{verbatim}
\begin{equation}
(\cdot), [\cdot], \{\cdot\}, |\cdot|, 
\lVert\cdot\rVert, \langle\cdot\rangle, 
\lfloor\cdot\rfloor, \lceil\cdot\rceil
\end{equation}
  \end{verbatim} 
  \column{.5\textwidth}
  \begin{fmpage}{\textwidth}
\begin{equation}
(\cdot), [\cdot], \{\cdot\}, |\cdot|, \lVert\cdot\rVert, \langle\cdot\rangle, \lfloor\cdot\rfloor, \lceil\cdot\rceil
\end{equation}
  \end{fmpage}
  \end{columns}
\end{frame}


\begin{frame}[fragile]
\frametitle{Documento}
\framesubtitle{Fórmulas}
  \scriptsize
  \begin{columns}[c]
  \column{.5\textwidth}
  \begin{verbatim}
\begin{equation}
\frac{n!}{k!(n-k)!} = \binom{n}{k}
\end{equation}
  \end{verbatim}
  \column{.5\textwidth}
  \begin{fmpage}{\textwidth}
\begin{equation}
\frac{n!}{k!(n-k)!} = \binom{n}{k}
\end{equation}
  \end{fmpage}
  \end{columns}


  \begin{columns}[c]
  \column{.5\textwidth}
  \begin{verbatim}
\begin{equation}
\frac{n!}{k!(n-k)!} = {n \choose k}
\end{equation}
  \end{verbatim}
  \column{.5\textwidth}
  \begin{fmpage}{\textwidth}
\begin{equation}
\frac{n!}{k!(n-k)!} = {n \choose k}
\end{equation}
  \end{fmpage}
  \end{columns}

  \begin{columns}[c]
  \column{.5\textwidth}
  \begin{verbatim}
\begin{equation}
\frac{\frac{1}{x}+\frac{1}{y}}{y-z}
\end{equation}
  \end{verbatim}
  \column{.5\textwidth}
  \begin{fmpage}{\textwidth}
\begin{equation}
\frac{\frac{1}{x}+\frac{1}{y}}{y-z}
\end{equation}
  \end{fmpage}
  \end{columns}
\end{frame}


\begin{frame}[fragile]
\frametitle{Documento}
\framesubtitle{Fórmulas}
  \scriptsize
  \begin{columns}[c]
  \column{.5\textwidth}
  \begin{verbatim}
\begin{equation}
  x = a_0 + \cfrac{1}{a_1
          + \cfrac{1}{a_2
          + \cfrac{1}{a_3 + a_4}}}
\end{equation}
  \end{verbatim}
  \column{.5\textwidth}
  \begin{fmpage}{\textwidth}
\begin{equation}
  x = a_0 + \frac{1}{a_1 + \frac{1}{a_2 + \frac{1}{a_3 + a_4}}}
\end{equation}
  \end{fmpage}
  \end{columns}


  \begin{columns}[c]
  \column{.5\textwidth}
  \begin{verbatim}
\begin{equation}
\frac{
    \begin{array}[b]{r}
      \left( x_1 x_2 \right)\\
      \times \left( x'_1 x'_2 \right)
    \end{array}
  }{
    \left( y_1y_2y_3y_4 \right)
  }
\end{equation}
  \end{verbatim}
  \column{.5\textwidth}
  \begin{fmpage}{\textwidth}
\begin{equation}
\frac{
    \begin{array}[b]{r}
      \left( x_1 x_2 \right)\\
      \times \left( x'_1 x'_2 \right)
    \end{array}
  }{
    \left( y_1y_2y_3y_4 \right)
  }
\end{equation}
  \end{fmpage}
  \end{columns}
\end{frame}


\begin{frame}[fragile]
\frametitle{Documento}
\framesubtitle{Fórmulas}
  \scriptsize
  \begin{columns}[c]
  \column{.5\textwidth}
  \begin{verbatim}
\begin{equation}
\sqrt[n]{1+x+x^2+x^3+\ldots}
\end{equation}
  \end{verbatim}
  \column{.5\textwidth}
  \begin{fmpage}{\textwidth}
\begin{equation}
\sqrt[n]{1+x+x^2+x^3+\ldots}
\end{equation}
  \end{fmpage}
  \end{columns}

  \begin{columns}[c]
  \column{.5\textwidth}
  \begin{verbatim}
\begin{equation}
\sum_{i=1}^{10} t_i
\end{equation}
  \end{verbatim}
  \column{.5\textwidth}
  \begin{fmpage}{\textwidth}
\begin{equation}
\sum_{i=1}^{10} t_i
\end{equation}
  \end{fmpage}
  \end{columns}

  \begin{columns}[c]
  \column{.5\textwidth}
  \begin{verbatim}
\begin{equation}
\int_0^\infty e^{-x}\,\mathrm{d}x
\end{equation}
  \end{verbatim}
  \column{.5\textwidth}
  \begin{fmpage}{\textwidth}
\begin{equation}
\int_0^\infty e^{-x}\,\mathrm{d}x
\end{equation}
  \end{fmpage}
  \end{columns}
\end{frame}


\begin{frame}[fragile]
\frametitle{Documento}
\framesubtitle{Fórmulas}
  \scriptsize
  \begin{columns}[c]
  \column{.5\textwidth}
  \begin{verbatim}
\begin{equation}
\sum_{\substack{
   0<i<m \\
   0<j<n
  }}
 P(i,j)
\end{equation}
  \end{verbatim}
  \column{.5\textwidth}
  \begin{fmpage}{\textwidth}
\begin{equation}
\sum_{\substack{
   0<i<m \\
   0<j<n
  }}
 P(i,j)
\end{equation}
  \end{fmpage}
  \end{columns}

  \begin{columns}[c]
  \column{.5\textwidth}
  \begin{verbatim}
\begin{equation}
\int\limits_a^b
\end{equation}
  \end{verbatim}
  \column{.5\textwidth}
  \begin{fmpage}{\textwidth}
\begin{equation}
\int\limits_a^b
\end{equation}
  \end{fmpage}
  \end{columns}

  \begin{columns}[c]
  \column{.5\textwidth}
  \begin{verbatim}
\begin{equation}
\prod \bigoplus \bigotimes 
\bigcup \bigcap \oint \iint \iiint
\end{equation}
  \end{verbatim}
  \column{.5\textwidth}
  \begin{fmpage}{\textwidth}
\begin{equation}
\prod \bigoplus \bigotimes 
\bigcup \bigcap \oint \iint \iiint
\end{equation}
  \end{fmpage}
  \end{columns}
\end{frame}


\begin{frame}[fragile]
\frametitle{Documento}
\framesubtitle{Fórmulas}
  \scriptsize
  \begin{columns}[c]
  \column{.5\textwidth}
  \begin{verbatim}
\begin{equation}
\left(\frac{x^2}{y^3}\right)
\end{equation}
  \end{verbatim}
  \column{.5\textwidth}
  \begin{fmpage}{\textwidth}
\begin{equation}
\left(\frac{x^2}{y^3}\right)
\end{equation}
  \end{fmpage}
  \end{columns}


  \begin{columns}[c]
  \column{.5\textwidth}
  \begin{verbatim}
\begin{equation}
\left\{\frac{x^2}{y^3}\right\}
\end{equation}
  \end{verbatim}
  \column{.5\textwidth}
  \begin{fmpage}{\textwidth}
\begin{equation}
\left\{\frac{x^2}{y^3}\right\}
\end{equation}
  \end{fmpage}
  \end{columns}

  \begin{columns}[c]
  \column{.5\textwidth}
  \begin{verbatim}
\begin{equation}
\left.\frac{x^3}{3}\right|_0^1
\end{equation}
  \end{verbatim}
  \column{.5\textwidth}
  \begin{fmpage}{\textwidth}
\begin{equation}
\left.\frac{x^3}{3}\right|_0^1
\end{equation}
  \end{fmpage}
  \end{columns}
\end{frame}


\begin{frame}[fragile]
\frametitle{Documento}
\framesubtitle{Fórmulas}
  \scriptsize
  \begin{columns}[c]
  \column{.5\textwidth}
  \begin{verbatim}
\begin{equation}
\begin{matrix}
  a & b & c \\
  d & e & f \\
  g & h & i
 \end{matrix}
\end{equation}
  \end{verbatim}
  \column{.5\textwidth}
  \begin{fmpage}{\textwidth}
\begin{equation}
\begin{matrix}
  a & b & c \\
  d & e & f \\
  g & h & i
 \end{matrix}
\end{equation}
  \end{fmpage}
  \end{columns}

  \begin{columns}[c]
  \column{.5\textwidth}
  \begin{verbatim}
\begin{equation}
A_{m,n} =
\begin{pmatrix}
a_{1,1} & a_{1,2} & \cdots & a_{1,n} \\
a_{2,1} & a_{2,2} & \cdots & a_{2,n} \\
\vdots  & \vdots  & \ddots & \vdots  \\
a_{m,1} & a_{m,2} & \cdots & a_{m,n}
\end{pmatrix}
\end{equation}
  \end{verbatim}
  \column{.5\textwidth}
  \begin{fmpage}{\textwidth}
\begin{equation}
 A_{m,n} =
 \begin{pmatrix}
  a_{1,1} & a_{1,2} & \cdots & a_{1,n} \\
  a_{2,1} & a_{2,2} & \cdots & a_{2,n} \\
  \vdots  & \vdots  & \ddots & \vdots  \\
  a_{m,1} & a_{m,2} & \cdots & a_{m,n}
 \end{pmatrix}
\end{equation}
  \end{fmpage}
  \end{columns}

%  \begin{columns}[c]
%  \column{.5\textwidth}
%  \begin{verbatim}
%\begin{equation}
%M = \bordermatrix{~ & x & y \cr
%                  A & 1 & 0 \cr
%                  B & 0 & 1 \cr}
%\end{equation}
%  \end{verbatim}
%  \column{.5\textwidth}
%  \begin{fmpage}{\textwidth}
%\begin{equation}
%M = \bordermatrix{~ & x & y \cr
%                  A & 1 & 0 \cr
%                  B & 0 & 1 \cr}
%\end{equation}
%  \end{fmpage}
%  \end{columns}
\end{frame}


\begin{frame}[fragile]
\frametitle{Documento}
\framesubtitle{Fórmulas}
  \scriptsize
  \begin{columns}[c]
  \column{.5\textwidth}
  \begin{verbatim}
\begin{equation}
f(n) = \left\{ 
\begin{array}{l l}
n/2 & \quad \text{if $n$ is even}\\
-(n+1)/2 & \quad \text{if $n$ is odd}\\
\end{array} \right.
\end{equation}
  \end{verbatim}
  \column{.5\textwidth}
  \begin{fmpage}{\textwidth}
\begin{equation}
f(n) = \left\{ 
\begin{array}{l l}
n/2 & \quad \text{if $n$ is even}\\
-(n+1)/2 & \quad \text{if $n$ is odd}\\
\end{array} \right.
\end{equation}
  \end{fmpage}
  \end{columns}

  \vspace{-0.3cm}
  \begin{columns}[c]
  \column{.5\textwidth}
  \begin{verbatim}
\begin{eqnarray*}
\cos 2\theta & = & \cos^2 \theta - 
                   \sin^2 \theta \\
             & = & 2 \cos^2 \theta - 1.
\end{eqnarray*}
  \end{verbatim}
  \column{.5\textwidth}
  \begin{fmpage}{\textwidth}
\begin{eqnarray*}
\cos 2\theta & = & \cos^2 \theta - 
                   \sin^2 \theta \\
             & = & 2 \cos^2 \theta - 1.
\end{eqnarray*}
  \end{fmpage}
  \end{columns}

 \vspace{-0.3cm}
  \begin{columns}[c]
  \column{.5\textwidth}
  \begin{verbatim}
\begin{align*}
  z_0 &= d = 0 \\
  z_{n+1} &= z_n^2+c
\end{align*}
  \end{verbatim}
  \column{.5\textwidth}
  \begin{fmpage}{\textwidth}
\begin{align*}
  z_0 &= d = 0 \\
  z_{n+1} &= z_n^2+c
\end{align*}
  \end{fmpage}
  \end{columns}
\end{frame}


\begin{frame}[fragile]
\frametitle{Documento}
\framesubtitle{Fórmulas}
\href{http://tug.ctan.org/info/short-math-guide/short-math-guide.pdf}{Short Math Guide for \LaTeX}

\url{https://www.overleaf.com/learn/latex/Mathematical_expressions}
\url{https://en.wikibooks.org/wiki/LaTeX/Mathematics}
\url{https://en.wikibooks.org/wiki/LaTeX/Advanced_Mathematics}
\url{https://en.wikibooks.org/wiki/LaTeX/Theorems}
\end{frame}
