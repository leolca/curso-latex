\begin{frame}[fragile]
\frametitle{Documento}
\framesubtitle{Bibliografia - como inserir uma obra e citá-la}
  \scriptsize
  \begin{columns}[c]
  \column{.5\textwidth}
  \begin{verbatim}
  @book{Knuth86,
  author    = {Donald E. Knuth},
  title     = {The TeXbook},
  publisher = {Addison-Wesley},
  year      = {1986},
  isbn      = {0-201-13447-0}
  }
  
  Citação no texto \cite{Knuth86}, \citep{Knuth86}.
  \end{verbatim} 
  \column{.5\textwidth}
  \begin{fmpage}{\textwidth}
  Citação no texto \cite{Knuth86}, \citep{Knuth86}.
  \end{fmpage}
  \end{columns}
\end{frame}


\begin{frame}[fragile]
\frametitle{Documento}
\framesubtitle{Atributos de um item de bibliografia}
  \scriptsize
  \begin{columns}[c]
  \column{.5\textwidth}
  \begin{verbatim}
@article{...,
    author  = "...",
    title   = "...",
    year    = "...",
    journal = "...",
    volume  = "...",
    number  = "...",
    pages   = "..."
}
  \end{verbatim} 
  \column{.5\textwidth}
\begin{verbatim}
@conference{...,
    author    = "...",
    title     = "...",
    booktitle = "...",
    %editor   = "...",
    %volume   = "...",
    %number   = "...",
    %series   = "...",
    %pages    = "...",
    %address  = "...",
    year      = "...",
    %month    = "...",
    %publisher= "...",
    %note     = "..."
}
\end{verbatim}
  \end{columns}
\end{frame}


\begin{frame}[fragile]
\frametitle{Documento}
\framesubtitle{Bibliografia - classes dos itens}
  \scriptsize
  \begin{columns}[c]
  \column{.5\textwidth}
  \begin{verbatim}
@inbook

@incollection

@inproceedings 

@mastersthesis
  \end{verbatim} 
  \column{.5\textwidth}
\begin{verbatim}
@misc 

@phdthesis 

@proceedings 

@techreport 

@unpublished 
\end{verbatim}
  \end{columns}
\end{frame}


\begin{frame}[fragile]
\frametitle{Documento}
\framesubtitle{Bibliografia - estilo}
  \scriptsize
  \begin{verbatim}
\bibliographystyle{apalike}
\bibliography{bibliografia}
  \end{verbatim}
  
  ver slide \pageref{bibliografia}.
\end{frame}


