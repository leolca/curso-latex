\begin{frame}[fragile]
\frametitle{Comandos}
\framesubtitle{definindo novos comandos}

\begin{verbatim}
\newcommand{\R}{$\mathbb{R}$}
\end{verbatim}

Podemos definir novos comandos: \R.
É uma boa prática definí-los no preambulo do documento.

\end{frame}

\begin{frame}[fragile]
\frametitle{Comandos}
\framesubtitle{comandos com parâmetros}

\begin{verbatim}
\newcommand{\bb}[1]{$\mathbb{#1}$}

utilização:
\bb{C}, \bb{B}, \bb{D}
\end{verbatim}

Definimos acima um comando que possui um parâmetro.
Pode assim fácilmente gerar: \bb{C}, \bb{B}, \bb{D}.

\end{frame}

